\section{System Design}

\subsection{System Architectures}

  The system contains two different type of databases. The first databases \textbf{PlayerDB}
  combines with \textbf{TrustedDB} and \textbf{UntrustedDB} where
  presistent the player inputs whether the overall result is reliable or not. 
  A reliable player shall pass the system \textbf{Player Rating Model}. 
  Once the task result from new player is reliable, then the system will
  reuse the player input into our \textbf{Disaster Evalutation Model} and presistent it in the second
  database \textbf{ResultDB}. Stakeholder make querys to this monitoring database. 
  Figure \ref{fig:arch} illustrate the overall disaster system design.

    \begin{figure}[htp]
    \centering
    \includegraphics[width=\textwidth]{figures/system2}
    \caption{System Design Overview}
    \label{fig:arch}
    \end{figure}

  The most important issue of this human computation system is to solve weather the input of player 
  is trustable or not. To solve this problem, we designed a \textbf{Task Generator} that combines 
  trusted results and seperate new satellite images to players. 

\subsection{System Components}

\subsubsection{Database Fields}

  For the convinience of model establishment, we describe the system database PlayerDB 
  Fields as well as the fields of database ResultDB in the follows listing \ref{lst:playerdb}
  and \ref{lst:resultdb}:

\noindent\begin{minipage}{.45\textwidth}
\begin{lstlisting}[
    caption={Player Database Fields},
    label={lst:playerdb}
]
[
 {
  "anonymous_id": number,
  "reliable": boolean,
  "trust_value": number
  "tasks": [
   {
    "image": image_path,
    "at_time": time, 
    "ROI": [
     {
      "latitude": number,
      "longitude": number,
      "tags": [tag1, tag2, ...]
     }, ...
    ]
   }
  ]
 }, ...
]
\end{lstlisting}
\end{minipage}\hfill
\begin{minipage}{.45\textwidth}
\begin{lstlisting}[
  caption={Results Database Fields},
  label={lst:resultdb}
]
[
 {
  "area_id": number,
  "disaster_level": number,
  "history": [
   {
    "at_time": time,
    "image": image_path,
    "ROI": [
     {
      "latitude": number,
      "longitude": number,
      "tags": [tag1, tag2, ...]
    }
   ]
  }, 
  ...
 ]
}, ...
]
\end{lstlisting}
\end{minipage}

  In this disaster monitoring system, our participant do not need to register an account,
  and the system shall assign a anonymous\_id for each participant which significantly 
  accelerate 

  Then we give some basic definition for the system model.

  \begin{definition}
  \label{def:roi}
  The \textbf{Region of Interests (ROI)} $ROI_i$ is a player selected area of player $i$.
  \end{definition}

  \begin{definition}
  \label{def:tagv}
  The \textbf{tags vector} $T_i$ of player $i$ is indicated by a vector where the components represent by the count of all tags:
  \[
    T_i = (|\text{tag}_1|, |\text{tag}_2|, ..., |\text{tag}_n|)
  \]
  where 
  \begin{itemize}
  \item $n$ is the number of current exist tags;
  \item $|\text{tag}_n|$ is the occurrance of $\text{tag}_n$.
  \end{itemize}
  \end{definition}
  For instance, there are 5 different tags $\text{tag}_1, \text{tag}_2, \text{tag}_3, \text{tag}_4, \text{tag}_5$ exist in the current system,
  player $i$ generates tags list $\{\text{tag}_1, \text{tag}_2, \text{tag}_3\}$, player $j$ generates tag list
  $\{\text{tag}_4, \text{tag}_4, \text{tag}_5\}$. Then $T_i$ of player $i$ is $(1, 1, 1, 0, 0)$ and $T_j$ of player $j$ is $(0, 0, 0, 2, 5)$.

  \begin{definition}
  \label{def:weightv}
  The \textbf{weight vector} $v = (p(\text{tag}_1), p(\text{tag}_2), ..., p(\text{tag}_n))$ \textbf{of all tags} 
  can be calculated by the following equation \ref{ptag}:
  \begin{equation}
  \label{eq:ptag}
  p(\text{tag}_i) = \frac{|\text{tag}_i|}{\sum_{j=1}^{n}{|\text{tag}_j|}}
  \end{equation}
  where
  \begin{itemize}
  \item $n$ is the number of current exist tags;
  \item $|\text{tag}_i|$ is the occurrance of $\text{tag}_i$.
  \end{itemize}
  \end{definition}

\subsubsection{Player Task Generator}

  The \textbf{Player Task Generator (PTG)} combines images from satellite and ResultDB. 
  In the first step, as we discussed before, to solve the imformation leakage problem,
  PTG shall split a monitoring region into $m\times n$ small pieces of images, and also assign a 
  unique \textbf{areaID} for each pieces, i.e. $(\text{areaID}, \text{time})$ 
  specifice a unique image for user tasks. 

  The second generate step is to retrieve tagged images from \textbf{ResultDB}. Then combine
  all images as a user task assign to a new upcomming player. Each user task contains 
  half of untagged images and half of tagged images.

  In short, The Data Model (only ouput here) for PTG is:

  \[
  \{(\text{areaID}_1, \text{time}_1), ..., (\text{areaID}_n, \text{time}_n)\}
  \]

  with $\text{areaID}_1$ to $\text{areaID}_{\floor{n}}$ are from satellite and 
  $\text{areaID}_{\floor{n}+1}$ to $\text{areaID}_{n}$ are from \textbf{ResultDB}.

\subsubsection{Player Rating Model}

  This subsection describes the Player Rating Model inside our Disaster Monitroing system.
  PageRank was first proposed by Lary Page \cite{page1999pagerank} and applied to social analysis in \cite{bonacich2001eigenvector}. 
  It is commonly used for expressing the stability of physical systems and the relative importance, 
  so-called centralities, of the nodes of a network. We transfer the basic idea of centralities 
  and use eigenvalue as a \textbf{Trust Value (TV)} for each players to distinguish manicious players.

  Considering a partial fully connected directed graph betwen players. 
  Each player is a node of the Player Rating Graph (PRG) as illustrate in figure \ref{fig:graph}.

  \begin{figure}[htp]
  \centering
  \includegraphics[width=0.3\columnwidth]{figures/graph}
  \caption{Player Rating Model}
  \label{fig:graph}
  \end{figure}
  
  To define the edge weight, according to the database feild design of a player, each player
  output ROIs for each task region of a player task, and each ROI contains a tags list, thus, 
  one can use three festures: $\text{ROI}, \text{tags}, TV$.

  \begin{definition}
  The weight from player $i$ to player $j$ can be formalized as follows formula \ref{eq:weight}:
  \begin{equation}
  \label{eq:weight}
  w_{ij} = 
  \sum_{\text{ROI}\in\text{ROIs}}{
    \text{TV}_i \times
    \frac{\text{ROI}_i\cap\text{ROI}_j}{\text{ROI}_i}
    \left( 2-\frac{Cov(T_i, T_j; v)}
        {Cov(T_i, T_i; v)\times Cov(T_j, T_j; v)} \right)
  }
  \end{equation}

  where 
  
  \begin{itemize}
    \item $TV_i$ is the trust value of player $i$;
    \item $\text{ROI}_i$ is the selected ROI from player $i$;
    \item $T_i$ is the tags vector of player $i$;
    \item $Cov(x, y; v)$ is the weighted covariance of $x$ and $y$ via $v$;
    \item $v$ is the weight vector of all tags.
  \end{itemize}
  \end{definition}

  The first part of the definition $\sum_{\text{ROI}\in\text{ROIs}}$ summarized all possible 
  ROI between player $i$ and player $j$. The theioretically item of this formular is the number
  of ROI from player $i$ multiply the number of ROI from player $j$. Nevertheless, it can be
  significantlly decreased in this particular scienario. Considering player i and player j 
  with two ROIs as illustrate in figure \ref{fig:performance}.

  \begin{figure}[htp]
  \centering
  \includegraphics[width=0.5\columnwidth]{figures/performance}
  \caption{Two players with two ROIs}
  \label{fig:performance}
  \end{figure}

  One can expand equation \ref{eq:weight} as follows formula \ref{eq:expand}:

  \begin{multline}
  \label{eq:expand}
  w_{ij} = \text{TV}_i \times \left(2-\frac{Cov(T_i, T_j; v)}{Cov(T_i, T_i; v)\times Cov(T_j, T_j; v)}\right) \times \\
    \left( \frac{\text{ROI}_i^1\cap\text{ROI}_j^1}{\text{ROI}_i^1}           
    + \frac{\text{ROI}_i^1\cap\text{ROI}_j^2}{\text{ROI}_i^1}           
    + \frac{\text{ROI}_i^2\cap\text{ROI}_j^1}{\text{ROI}_i^2}           
    + \frac{\text{ROI}_i^2\cap\text{ROI}_j^1}{\text{ROI}_i^2} \right)
  \end{multline}

  Fortunately, the second and the third part of the expation are equal to zero.

  We call the second part $\text{TV}_i \times \frac{\text{ROI}_i\cap\text{ROI}_j}{\text{ROI}_i}$ 
  of formula \ref{eq:weight} as \textbf{Matching Area Retio (MAR)}. 
  It was inspired by a common computer vision criteria,
  the so called Intersection over Union (IoU), also called Jaccard Index in mathematics\cite{real1996probabilistic},
  which is the standard performance measure that is commonly used for the object category segmentation problem.
  Nevertheless, MAR is not equal to the IoU of ROIs of player $i$  and player $j$ since
  it only use the ROI of player $i$ as denominator instead of the union of ROIs of player $i$ and player $j$,
  which leads the differnce between MAR and IoU. There are two reason to use MAR instead of IoU:
  Firstly, IoU as weight of graph causes the directed graph to an undirected graph due to the IoU of player $i$ to $j$
  is as same as the IoU of player $j$ to $i$; Furthermore, player $i$ as the evaluator from $i$ to $j$ 
  should be the performance base.

  The third part $\frac{Cov(T_i, T_j; v)}{Cov(T_i, T_i; v)\times Cov(T_j, T_j; v)}$
  of formula \ref{eq:weight} is applied by Weighted Pearson Correlation Coefficient. 
  
  To calculate the eigenvalue of the ajacency matrix of PRG, one can use the normalized ajacency matrix 
  through the following formula \ref{eq:normalize}:
  \begin{equation}
  \label{eq:normalize}
  A = (a_{ij}) = (\frac{w_{ij}}{\sum_{j}{w_{ij}}})\\
  \end{equation}

  \begin{theorem}
  Matrix A is irreducible, real, non-negative, column-stochastic, and diagonal element being positive.
  \end{theorem}

  \begin{proof}
  \textbf{Irreducibility}: A is normalized through an ajacency matrix of a strong connected player
  rating graph, which proofs A is irreducible.

  \textbf{Real elements}: trivial.
  
  \textbf{Non-negative elements}: We only need to prove $\text{TV}_i$, 
    $\frac{\text{ROI}_i\cap\text{ROI}_j}{\text{ROI}_i}$ and 
    $2-\frac{Cov(T_i, T_j; v)}{Cov(T_i, T_i; v)\times Cov(T_j, T_j; v)}$ are non-negative 
    respectively. $\text{TV}_i$ is the eigenvalues of normalized graph ajacency matrix, 
    thus the codomain of $\text{TV}_i$ lies $(0, 1\rbrack$; For MAR, its range is obviously from 0 to 1,
    which lies $\lbrack 0, 1 \rbrack$; For $2-\frac{Cov(T_i, T_j; v)}{Cov(T_i, T_i; v)\times Cov(T_j, T_j; v)}$,
    the Pearson Correlation lies on $[-1, 1]$, then this part lies on $[1, 3]$.
    Three parts are non-negative.
  
  \textbf{Positive diagonal elements}: The diagonal elements can be formalized by follows:

  \[
  w_{ii} = 
  \sum_{\text{ROI}\in\text{ROIs}}{
    \text{TV}_i \times
    \frac{\text{ROI}_i\cap\text{ROI}_i}{\text{ROI}_i}
    \left( 2-\frac{Cov(T_i, T_i; v)}
        {Cov(T_i, T_i; v)\times Cov(T_i, T_i; v)} \right)
  } = \sum_{\text{ROI}\in\text{ROIs}}{\text{TV}_i} > 0
  \]
  
  \textbf{Column stochastic}: according to the definition of matrix $A$, the sum of the column
  elements is:
  \[
    \sum_{i}{\frac{w_{ij}}{\sum_{j}{w_{ij}}}} 
    = \frac{\sum_{i}{w_{ij}}}{\sum_{j}{w_{ij}}} = 1
  \]
  \end{proof}

  We has proved the existence and uniqueness of eigenvalues of normalized PRG ajacency matrix, 
  one can use the corresponding eigenvalues to represent the trust value of players. Thus, we have:

  \begin{definition}
  A \textbf{Trust Value} $TV_i$ of player $i$ represents by the $i$-th eigenvalue of normalized PRG ajacency matrix $A$
  \end{definition}

  This definition can represents the rating score from $i$ to $j$.
  
  Note that:

  \begin{itemize}
    \item $(anonymous\_id, image, event\_time, ROI)$ is the primary key of the input vector;
    \item A player can generate multiple vectors to rating system even for same image;
    \item The event\_time is the capture time of the satellite image.
  \end{itemize}

  Terefore eigenvalue of A is the player trust value. When a new player tagging task need to be rated,

  \begin{itemize}
    \item which means we need introduce a new node to the graph
    \item need calculate the trust value of new graph
    \item let $t’$ is the trust value of new player
    \item if $t’ >= \text{mean}(\text{old\_eigenvalues})$, then it is a reliable player, otherwise drop it.
  \end{itemize}

  In short, the input and output Data Model of PRM are as follows. For input:

  \[
    (\text{anonymous\_id}, \text{area\_id}, \text{time}, \text{ROIs}, \text{tags})
  \]

  For model output: 
  
  \[
  (\text{anonymous\_id}, \text{TV})
  \]

  \subsubsection{Disaster Evaluation Model}

  System like ESP\cite{von2004labeling}, ARTigo\cite{wieser2013artigo} has proved that 
  human inputs are valuable and useful.

    For an area at time t, define disaster level as follows:

    \[
    v_{area} = \frac{
    \sum_{\text{tag}\in\text{tags}}
      {w_{tag}\times\#(\text{tag})}
    }
    {\sum_{area\in areas}{\sum_{\text{tag}\in\text{tags}}{w_{tag}\times\#(\text{tag})}}}
    \]

    where $w_{tag}$ is pre-defined weight by system, $\#(tag)$ is the occur number of a tag.

    Return value:

    \begin{itemize}
      \item disaster region: $\cup_{ROI\in ROIs}{ROI}$
      \item disaster level: $v_{area}$
    \end{itemize}


    For query input:

    \[
    (\text{time}) or (\text{area\_id})/(\text{area\_id}, \text{time})
    \]

    For model output:

    \[
    (\text{area\_id}, \text{time}, \text{disaster\_level})
    \]

    Note that:

    \begin{itemize}
      \item All results are evaluated from reliable tasks
      \item Evaluation Model generated by all reliable history
    \end{itemize}

    Now we have trusted results, each area has its tagging history.


\subsection{Problem Handling}

The system we proposed is able to handling most of the common problem appeared in HC system.

\subsubsection{Model Cold Start and Initialization}

\subsubsection{Malicious User Detection and Classificaition}

\subsection{Portabilities}


\subsection{Summary}

    \begin{itemize}
      \item Task Generator combines trusted results assign to players;
      \item Always treat player as new player, but integrated as old player if exists;
      \item Use ROI matching rate as graph edge weight, eigenvalue as trust value of player;
      \item Disaster Evaluation use pre-defined weight, then defined the disaster level
    \end{itemize}
