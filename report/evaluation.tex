\section{System Evaluation and Success Criteria}

\subsection{Evaluation and success criteria}
  \subsubsection{Model Evaluation}

  Malicious player detection is a classification problem. 
  One can generate random data and test the Rating Model through accuracy and recall, even ROC curve \cite{hanley1982meaning}.

  The click behavior has been researched for years and address by FFitts Law \cite{bi2013ffitts}.
  It modeled and proved the distribution of click behavior for a certain click goal point is a normal distribution.
  Thus, with probablistic view, the top left corner of ROI exists, then the user click selection 
  for this point should follows normal distribution, as shown in figure \ref{fig:evaluation}.

  \begin{figure}[htp]
  \centering
  \includegraphics[width=0.5\columnwidth]{figures/evaluation}
  \caption{Data Simulation}
  \label{fig:evaluation}
  \end{figure}

  Therefore, to generate ROIs, let $(x, y)$ is the player ROI start point,  $(H_{ROI}$, $W_{ROI})$ is the height
  and width pair of this ROI, then we generate the random dataset for these variables by a given
  parameter $\delta$: $(x, y) \sim (x+N(0, \delta), y+N(0, \delta)), (H_{ROI}, W_{ROI}) \sim (H_{ROI}+N(0, \delta), W_{ROI}+N(0, \delta))$.
  To generate tags, we propose randomly pick random number of tags.

  Then once can perform this random dataset on our system to evaluate the classification accuracy and recall rate to
  evaluate the overall performance of this system.

  \subsubsection{Issues on Social and Ethical Aspects}

\subsection{Limitation of the System}

\subsubsection{Evaluation Outdate}

A limitation occurs in our social network based model is each disaster level evaluation get invalid 
if the region image outdate. 
We assume the satellite monitors a region and take picutre between intervals. However, our evaluation
model only calculate the disaster level in a unique moment, which means the disaster level need 
transvaluation when a new image come out.
If our player are not enough so that the region images always have to wait new evaluation, then the
disaster level will never be calculated.

A possible solution is to consider the region disaster level history as a time series. Then we can apply
some prediction method for it. For instance, we have time series: $(t_1, t_2, t_3, ..., t_n)$
and its corresponding disaster level: $(DL_1, DL_2, DL_3, ..., DL_n)$.
Then we can use these time series to predict the disaster level at time $t_(n+1)$.

At the same time, we also have the historical data of trust value of a player. We can also
use time series prediction to predict the players trust value. But in all of these, the time series
of disaster level is not stationary but the time series of trust value is stationary.