\section{Conclusions and Future Works}

\subsection{Conclusions}

In this report, we proposed a comprehensive design of a human computation system for disaster monitoring,
and we discussed its mathematical foundations as well as the possible issues caused by this system,
then gives few options to solve these issues. 

In the chapter of \hyperref[chapter:func]{Functionalities}, 
we illustrated a mockup of GWAPs-based disaster monitoring
human computation system for game players as well as stakeholders, and then described its
necessary functions and interaction logic. On thoughts of system implementation, we discussed the possible technology stack of 
this system on the Web.

Afterward, in the chapter of \hyperref[chapter:design]{Design and Models}, we modeled the entire system theoretically in details 
that make sure it can run consistently. 
First of all, we defined a \textbf{\hyperref[chapter:prm]{Player Rating Model}} based on eingenvalue centralities
for calculating trust value of a player, and then we put forward an algorithm
that can be used in malicious user detection. As justification, we proved the correctness of this model.
Meanwhile, as the data aggregation, we transferred the problem of calculating disaster level of regions into
processing the expectancy of user tagging task inputs and proposed the \textbf{\hyperref[chapter:dem]{Disaster Evaluation Model}}.
Surely, we addressed the solution of cold start of the human computation system.
It is worth mentioning that the minimum initial trusted group under this scheme design only requires one person theoretically.

Furthermore, in the chapter of \hyperref[chapter:evaluation]{Evaluation}, we discussed theoretical evaluation criteria for this system,
and then declared the challenges and corresponding solutions for facing issues like data security, information leakage and loss, 
malicious detection as well as the lack of players.
Undoubtedly, the current system design still contains defects. 
Thus, we presented three analysis and possible improvements for evaluation outdated 
and gameplay playability. 

\subsection{Future Works}

Our system was described in general. We collect human inputs by ROI tagging tasks, 
which means any other human computation system that related to ROI tagging tasks can easily use this system backend design.
In addition, due to the fact that we do not have enough user inputs at present, 
we use a certainty algorithm instead of uncertainty probablistic-based algorithm to detect malicious groups. 
Considering malicious detection is classification problem, which seperate users into trusted groups and untrusted groups. 
One can apply any classification machine learning algorithms that are more suitable for the detection of malicious groups 
if our user input dataset is large enough.

Besides, as we mentioned before, the game player may encounter a situation that 
there is no ROI in some pictures which contain only landscapes like mountains, rivers and forests.
In this case, the game playability is significantly reduced. However, for future work,
we can filter out those images from our image database previously with collaborative computing of image recoginition technique,
so that one can collect more data from the game and make our image tagging game more efficiently.

\section*{Acknowledgements}
\addcontentsline{toc}{section}{\protect\numberline{}Acknowledgements}
The authors would like to thank Prof. Fran\c{c}ois Bry first for
his great and important suggestions on the model statements 
as well as information leak and loss problems of the disaster monitoring system;
we also thank Yingding Wang for his helpful discussions on system functionalities design 
and the model rationalizations as well as evaluations;
Finally, we also thank our schoolmate Huimin An for his inspiration of Bayesian perspective that
helps us handling human inputs with new tags successfully.

The resources of this project are open source on GitHub, such as paper \LaTeX~code, mockup drafts as well as
lab session beamer slides: \\
\url{https://github.com/changkun/hc-ss17-disaster-monitoring}.
